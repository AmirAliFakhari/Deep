\begin{frame}{The Goal: Minimizing Error}
    \framesubtitle{The Fundamental Task}
    \begin{itemize}
        \item The fundamental task in training a neural network is to find the optimal set of parameters (\bhighlight{weights W}) for a given architecture that minimizes the difference between the network's output and the true target values.
        \item This difference is quantified by a \bhighlight{cost function} (or loss function).
    \end{itemize}
\end{frame}

\begin{frame}{The Goal: Minimizing Error}
    \framesubtitle{The Setup}
    \begin{itemize}
        \item We are given a network architecture, which is a parametric function $f(x; W)$, and a set of training data $(x^{(n)}, y^{(n)})$.
        \item The total error $E(W)$ is the average loss over all N training instances:
        \[
            E(W) = \frac{1}{N} \sum_{n=1}^{N} \text{loss}(f(x^{(n)}; W), y^{(n)})
        \]
        \item Our goal is to find the weights $W$ that \bhighlight{minimize this cost function}.
    \end{itemize}
\end{frame}