\subsection{Adaptive Learning Rate Methods}

\begin{frame}{The Optimizer's Dilemma}
    \framesubtitle{One Learning Rate to Rule Them All?}
    \begin{itemize}
        \item We've seen that Momentum helps us move faster by building velocity.
        \item \textbf{But there's a problem:} It's like driving a car where all four wheels are forced to turn at the same angle. On a winding road (our ill-conditioned error surface), this is clumsy.
        \item Some parameters (wheels) need to turn sharply (small learning rate), while others need to go straight (large learning rate).
        \item \textbf{The Goal:} Give each parameter its own "smart" learning rate that adapts automatically to the terrain it's on.
    \end{itemize}
\end{frame}

\begin{frame}{Adagrad: The Historian}
    \framesubtitle{Every Gradient Leaves a Mark}
    \bhighlight{Adagrad} was a pioneering idea that gave each parameter a personal learning rate.
    \begin{itemize}
        \item \textbf{The Strategy:} Keep a running sum of the squares of all past gradients for each parameter.
        \item Parameters that have seen large gradients in the past will have their learning rate aggressively decreased.
        \item The update rule:
            $$ s^{(k)} = s^{(k-1)} + (\nabla_w E)^2 $$
            $$ W^{(k+1)} = W^{(k)} - \frac{\eta}{\sqrt{s^{(k)} + \epsilon}} \nabla_w E $$
        \item \textbf{The Fatal Flaw:} The denominator, $s^{(k)}$, is a sum that only ever grows, it eventually stops the learning process entirely as the learning rate vanishes.
    \end{itemize}
\end{frame}

\begin{frame}{RMSProp: The Forgetful Historian}
    \framesubtitle{Remembering Only the Recent Past}
    \bhighlight{RMSProp} solves Adagrad's problem by forgetting the distant past and focusing only on recent gradient history.
    \begin{itemize}
        \item Instead of a sum, it uses an \textbf{exponentially decaying moving average} of squared gradients.
        \item This prevents the denominator from growing indefinitely, allowing learning to continue.
        \item The update rule:
            $$ \overline{(\nabla_w E^2)}^{(k)} = \beta \overline{(\nabla_w E^2)}^{(k-1)} + (1-\beta)(\nabla_w E)^2 $$
            $$ W^{(k+1)} = W^{(k)} - \frac{\eta}{\sqrt{\overline{(\nabla_w E^2)}^{(k)} + \epsilon}} \nabla_w E $$
        \item It effectively normalizes each parameter's update by the magnitude of its recent gradients.
    \end{itemize}
\end{frame}

\begin{frame}{Adam: The Best of Both Worlds}
    \framesubtitle{Adaptive Moment Estimation}
    Adam is the most popular optimizer because it combines the best ideas we've seen so far.
    \begin{itemize}
        \item It's a hybrid of \textbf{Momentum} and \textbf{RMSProp}.
        \item It uses a moving average of the gradient itself (like Momentum) to track the direction of travel (the \textit{first moment}).
        \item It uses a moving average of the squared gradient (like RMSProp) to adapt the learning rate for each parameter (the \textit{second moment}).
    \end{itemize}
    \begin{alertblock}{The Result}
        An optimizer that knows both where it's going and how fast it should get there, for every single parameter.
    \end{alertblock}
\end{frame}

\begin{frame}{The Adam Update Rule}
    \framesubtitle{Putting It All Together}
    \small
    Adam maintains two moving averages, $m$ (for momentum) and $v$ (for variance scaling), and includes a bias-correction step.
    \begin{enumerate}
        \item First moment (Momentum):
            $$ m^{(k)} = \beta_1 m^{(k-1)} + (1-\beta_1) g^{(k)} $$
        \item Second moment (RMSProp):
            $$ v^{(k)} = \beta_2 v^{(k-1)} + (1-\beta_2) (g^{(k)})^2 $$
        \item Bias Correction (to counteract initialization at zero):
            $$ \hat{m}^{(k)} = \frac{m^{(k)}}{1 - \beta_1^k} \quad , \quad \hat{v}^{(k)} = \frac{v^{(k)}}{1 - \beta_2^k} $$
        \item Final Update Rule:
            $$ W^{(k+1)} = W^{(k)} - \frac{\eta}{\sqrt{\hat{v}^{(k)} + \epsilon}} \hat{m}^{(k)} $$
    \end{enumerate}
    \footnotesize{Common hyperparameters: $\eta$ (needs tuning), $\beta_1=0.9$, $\beta_2=0.999$, $\epsilon=10^{-8}$.}
\end{frame}

\begin{frame}{The Problem of Bias in Adam}
    \framesubtitle{The "Cold Start" Problem}
    \begin{itemize}
        \item Adam's moment vectors, $m$ and $v$, are initialized to zero.
        \item In the first few iterations, the moving averages are heavily weighted by these initial zeros, making them biased towards zero.
        \item This "cold start" causes the optimizer to take very small steps at the beginning of training, when it should be making the most progress.
    \end{itemize}
    \begin{alertblock}{The Solution: Bias Correction}
        The bias correction step divides the moment estimates by a factor that is initially small and approaches 1 over time. This scales up the early, biased estimates, giving the optimizer a "warm start" and allowing it to take meaningful steps from the very beginning.
    \end{alertblock}
\end{frame}

\begin{frame}{A Word of Caution}
    \framesubtitle{Is Newer Always Better?}
    \begin{itemize}
        \item While Adam is a fantastic default, it's not a silver bullet.
        \item Recent research has shown that adaptive methods can sometimes converge to "sharper" minima that generalize more poorly than the "flatter" minima found by well-tuned \bhighlight{SGD with Momentum}.
        \item There is no single optimizer that dominates across all possible problems.
        \item \textbf{The Takeaway:} Adam is a great starting point, but if you're aiming for state-of-the-art results, it can be worth trying to carefully tune SGD+Momentum as well.
    \end{itemize}
\end{frame}
